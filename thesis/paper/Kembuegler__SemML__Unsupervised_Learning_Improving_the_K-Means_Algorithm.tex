\documentclass[10pt,twocolumn,letterpaper]{article}

\usepackage{semml}
\usepackage{times}
\usepackage{epsfig}
\usepackage{graphicx}
\usepackage{amsmath}
\usepackage{amssymb}

% Include other packages here, before hyperref.

% If you comment hyperref and then uncomment it, you should delete
% egpaper.aux before re-running latex.  (Or just hit 'q' on the first latex
% run, let it finish, and you should be clear).
\usepackage[breaklinks=true,bookmarks=false]{hyperref}

\semmlfinalcopy % *** Uncomment this line for the final submission

\def\semmlPaperID{****} % *** Enter the SemML Paper ID here
\def\httilde{\mbox{\tt\raisebox{-.5ex}{\symbol{126}}}}

% Pages are numbered in submission mode, and unnumbered in camera-ready
%\ifsemmlfinal\pagestyle{empty}\fi
\setcounter{page}{1}
\begin{document}

%%%%%%%%% TITLE
\title{Unsupervised Learning: Improving the K-Means Algorithm}

\author{Juri Kembügler\\
Friedrich-Alexander-Universität\\
Erlangen, Germany\\
{\tt\small\href{mailto:juri.kembuegler@fau.de}{juri.kembuegler@fau.de}}
% For a paper whose authors are all at the same institution,
% omit the following lines up until the closing ``}''.
% Additional authors and addresses can be added with ``\and'',
% just like the second author.
% To save space, use either the email address or home page, not both
%\and
%Second Author\\
%Institution2\\
%First line of institution2 address\\
%{\tt\small secondauthor@i2.org}
}

\maketitle
%\thispagestyle{empty}

%%%%%%%%% ABSTRACT
\begin{abstract}
    The ABSTRACT is to be in fully-justified italicized text, at the top
    of the left-hand column, below the author and affiliation
    information. Use the word ``Abstract'' as the title, in 12-point
    Times, boldface type, centered relative to the column, initially
    capitalized. The abstract is to be in 10-point, single-spaced type.
    Leave two blank lines after the Abstract, then begin the main text.
    Look at previous SemML abstracts to get a feel for style and length.
\end{abstract}

%%%%%%%%% BODY TEXT

\section{Introduction}\label{sec:introduction}

Digitalisation has permeated various aspects of our daily lives, from sports
and fitness tracking to entertainment platforms such as YouTube and Twitch, as
well as digital newspaper consumption. A key consequence of this transformation
is the vast amount of data generated by computers, smartphones, smartwatches,
and other IoT devices. Extracting meaningful insights from this data requires
efficient and robust algorithms.

To extract meaningful insights from this data, Machine Learning has become an
essential tool. Machine Learning algorithms enable systems to identify patterns
and make decisions based on data. One important branch of Machine Learning is
unsupervised learning, where algorithms uncover hidden structures in data
without predefined labels. A fundamental technique in this domain is
clustering, with K-Means being one of the most widely used methods. However,
the performance of K-Means is highly dependent on the inital choice of
centroids.
% TODO: Source??? And maybe do not mention K-means and unsupervised learning herer?

\subsubsection{Contributions}

In this article, we make the following contributions:
%TODO: should we call this "contributions"?? Maybe

\begin{itemize}
    \item We provide an overview of clustering as an unsupervised Machine Learning
          technique.
    \item We analyze the K-Means algorithm, discussing its methodology and limitations.
    \item We present K-Means++ as a well-known and a statistical relatively new, less
          commonly used approach to improve K-Means.
    \item (Hopefully we also find an improvment for the statistical approach)
    \item We evaluate the performance of these improvements on various datasets.
\end{itemize}

%-------------------------------------------------------------------------

\section{Background and Related Work}\label{sec:background-and-related-work}

%------------------------------------------------------------------------

\subsection{Machine Learning techniques}\label{subsec:machine-learning-techniques}

Machine learning leverages statistical methods, mathematical models, and
numerical techniques to extract meaningful patterns from data, enabling tasks
such as summarization, clustering, visualization, and prediction. The learning
process is generally categorized into two main paradigms: supervised and
unsupervised learning. \cite{deuschle2019}

%-------------------------------------------------------------------------
% TODO: Check Plagiat on kushawahAndYadav2016
\subsubsection{Supervised Learning}\label{subsubsec:supervised-learning}

A supervised training technique is the technique where we consider both the
input and the output before using our machine learning model. We then attempt
to use our model to make statements or predictions about new unseen data.
\cite{deuschle2019} Known techniques are neural networks, several layers
perception and decision trees~\cite{kushawahAndYadav2016}.

%------------------------------------------------------------------------

\subsubsection{Unsupervised Learning}\label{subsubsec:unsupervised-learning}

An unsupervised training technique is the technique where we now only consider
a dataset of only the inputs. We then use our Machine Learining model, to
somewhat describe, summarize or categorize our data. \cite{deuschle2019} Known
techniques are non-identical types of clustering, amplitude and normalisation,
k-means and self-organising maps~\cite{kushawahAndYadav2016}.

%------------------------------------------------------------------------

\subsection{Clustering}\label{subsec:clustering}

Lorem ipsum dolor sit amet, consectetur adipiscing elit, sed do eiusmod tempor
incididunt ut labore et dolore magna aliqua. Ut enim ad minim veniam, quis
nostrud exercitation ullamco laboris nisi ut aliquip ex ea commodo consequat.
Duis aute irure dolor in reprehenderit in voluptate velit esse cillum dolore eu
fugiat nulla pariatur. Excepteur sint occaecat cupidatat non proident, sunt in
culpa qui officia deserunt mollit anim id est laborum.

%------------------------------------------------------------------------

\subsection{Common clustering algorithms}\label{subsec:common-clustering-algorithms}

Lorem ipsum dolor sit amet, consectetur adipiscing elit, sed do eiusmod tempor
incididunt ut labore et dolore magna aliqua. Ut enim ad minim veniam, quis
nostrud exercitation ullamco laboris nisi ut aliquip ex ea commodo consequat.

%------------------------------------------------------------------------

\subsection{Types of Clusters}\label{subsec:types-of-clusters}

Lorem ipsum dolor sit amet, consectetur adipiscing elit, sed do eiusmod tempor
incididunt ut labore et dolore magna aliqua. Ut enim ad minim veniam, quis
nostrud exercitation ullamco laboris nisi ut aliquip ex ea commodo consequat.
Duis aute irure dolor in reprehenderit in voluptate velit esse cillum dolore eu
fugiat nulla pariatur. Excepteur sint occaecat cupidatat non proident, sunt in
culpa qui officia deserunt mollit anim id est laborum.

%------------------------------------------------------------------------

\subsubsection{Well-separated clusters}

Lorem ipsum odor amet, consectetuer adipiscing elit. Tempor a proin quam cursus
tincidunt id fringilla. Netus hac suscipit maximus hendrerit luctus massa.
Hendrerit mollis nibh curabitur magnis netus ad diam. Sapien per conubia id
primis auctor. Porta metus feugiat tincidunt amet suscipit habitasse.

%------------------------------------------------------------------------

\subsubsection{Centre-based clusters}

Lorem ipsum odor amet, consectetuer adipiscing elit. Tempor a proin quam cursus
tincidunt id fringilla. Netus hac suscipit maximus hendrerit luctus massa.
Hendrerit mollis nibh curabitur magnis netus ad diam. Sapien per conubia id
primis auctor. Porta metus feugiat tincidunt amet suscipit habitasse.

%------------------------------------------------------------------------

\subsubsection{Contiguous clusters}

Lorem ipsum odor amet, consectetuer adipiscing elit. Tempor a proin quam cursus
tincidunt id fringilla. Netus hac suscipit maximus hendrerit luctus massa.
Hendrerit mollis nibh curabitur magnis netus ad diam. Sapien per conubia id
primis auctor. Porta metus feugiat tincidunt amet suscipit habitasse.

%------------------------------------------------------------------------

\subsubsection{Density-based clusters}

Lorem ipsum odor amet, consectetuer adipiscing elit. Tempor a proin quam cursus
tincidunt id fringilla. Netus hac suscipit maximus hendrerit luctus massa.
Hendrerit mollis nibh curabitur magnis netus ad diam. Sapien per conubia id
primis auctor. Porta metus feugiat tincidunt amet suscipit habitasse.

%------------------------------------------------------------------------

\section{K-Means}\label{sec:k-means}

Lorem ipsum dolor sit amet, consectetur adipiscing elit, sed do eiusmod tempor
incididunt ut labore et dolore magna aliqua. Ut enim ad minim veniam, quis
nostrud exercitation ullamco laboris nisi ut aliquip ex ea commodo consequat.
Duis aute irure dolor in reprehenderit in voluptate velit esse cillum dolore eu
fugiat nulla pariatur. Excepteur sint occaecat cupidatat non proident, sunt in
culpa qui officia deserunt mollit anim id est laborum.

%------------------------------------------------------------------------

\subsection{Procedure and Lloyd's algorithm}\label{subsec:procedure-and-lloyd's-algorithm}

Lorem ipsum odor amet, consectetuer adipiscing elit. Tempor a proin quam cursus
tincidunt id fringilla. Netus hac suscipit maximus hendrerit luctus massa.
Hendrerit mollis nibh curabitur magnis netus ad diam. Sapien per conubia id
primis auctor. Porta metus feugiat tincidunt amet suscipit habitasse.

%------------------------------------------------------------------------

\subsection{Problems and limitations}\label{subsec:problems-and-limitations}

Lorem ipsum odor amet, consectetuer adipiscing elit. Tempor a proin quam cursus
tincidunt id fringilla. Netus hac suscipit maximus hendrerit luctus massa.
Hendrerit mollis nibh curabitur magnis netus ad diam. Sapien per conubia id
primis auctor. Porta metus feugiat tincidunt amet suscipit habitasse.

%------------------------------------------------------------------------

\subsection{K-Means++}\label{subsec:k-means++}

Lorem ipsum odor amet, consectetuer adipiscing elit. Tempor a proin quam cursus
tincidunt id fringilla. Netus hac suscipit maximus hendrerit luctus massa.
Hendrerit mollis nibh curabitur magnis netus ad diam. Sapien per conubia id
primis auctor. Porta metus feugiat tincidunt amet suscipit habitasse.

%------------------------------------------------------------------------

\section{Statistically improving K-Means}\label{sec:statistically-improving-k-means}

Lorem ipsum dolor sit amet, consectetur adipiscing elit, sed do eiusmod tempor
incididunt ut labore et dolore magna aliqua. Ut enim ad minim veniam, quis
nostrud exercitation ullamco laboris nisi ut aliquip ex ea commodo consequat.
Duis aute irure dolor in reprehenderit in voluptate velit esse cillum dolore eu
fugiat nulla pariatur. Excepteur sint occaecat cupidatat non proident, sunt in
culpa qui officia deserunt mollit anim id est laborum.

%------------------------------------------------------------------------

\subsection{Identifying issues}\label{subsec:identifying-issues}

Lorem ipsum dolor sit amet, consectetur adipiscing elit, sed do eiusmod tempor
incididunt ut labore et dolore magna aliqua. Ut enim ad minim veniam, quis
nostrud exercitation ullamco laboris nisi ut aliquip ex ea commodo consequat.
Duis aute irure dolor in reprehenderit in voluptate velit esse cillum dolore eu
fugiat nulla pariatur. Excepteur sint occaecat cupidatat non proident, sunt in
culpa qui officia deserunt mollit anim id est laborum.

%------------------------------------------------------------------------

\subsection{The improved algorithm}\label{subsec:the-improved-algorithm}

Lorem ipsum dolor sit amet, consectetur adipiscing elit, sed do eiusmod tempor
incididunt ut labore et dolore magna aliqua. Ut enim ad minim veniam, quis
nostrud exercitation ullamco laboris nisi ut aliquip ex ea commodo consequat.
Duis aute irure dolor in reprehenderit in voluptate velit esse cillum dolore eu
fugiat nulla pariatur. Excepteur sint occaecat cupidatat non proident, sunt in
culpa qui officia deserunt mollit anim id est laborum.

%------------------------------------------------------------------------

\subsection{Challenges}\label{subsec:challenges}

Lorem ipsum dolor sit amet, consectetur adipiscing elit, sed do eiusmod tempor
incididunt ut labore et dolore magna aliqua. Ut enim ad minim veniam, quis
nostrud exercitation ullamco laboris nisi ut aliquip ex ea commodo consequat.
Duis aute irure dolor in reprehenderit in voluptate velit esse cillum dolore eu
fugiat nulla pariatur. Excepteur sint occaecat cupidatat non proident, sunt in
culpa qui officia deserunt mollit anim id est laborum.

%------------------------------------------------------------------------

\section{Results and Performance Evaluation}\label{sec:results-and-performance-evaluation}

Lorem ipsum dolor sit amet, consectetur adipiscing elit, sed do eiusmod tempor
incididunt ut labore et dolore magna aliqua. Ut enim ad minim veniam, quis
nostrud exercitation ullamco laboris nisi ut aliquip ex ea commodo consequat.
Duis aute irure dolor in reprehenderit in voluptate velit esse cillum dolore eu
fugiat nulla pariatur. Excepteur sint occaecat cupidatat non proident, sunt in
culpa qui officia deserunt mollit anim id est laborum.

%------------------------------------------------------------------------

\subsection{Metrics}

Lorem ipsum odor amet, consectetuer adipiscing elit. Tempor a proin quam cursus
tincidunt id fringilla. Netus hac suscipit maximus hendrerit luctus massa.
Hendrerit mollis nibh curabitur magnis netus ad diam. Sapien per conubia id
primis auctor. Porta metus feugiat tincidunt amet suscipit habitasse.

%------------------------------------------------------------------------

\subsection{Case Study: Mall Customer Segmentation}

Lorem ipsum odor amet, consectetuer adipiscing elit. Tempor a proin quam cursus
tincidunt id fringilla. Netus hac suscipit maximus hendrerit luctus massa.
Hendrerit mollis nibh curabitur magnis netus ad diam. Sapien per conubia id
primis auctor. Porta metus feugiat tincidunt amet suscipit habitasse.

%------------------------------------------------------------------------

\subsection{Measurements on different datasets}

Lorem ipsum odor amet, consectetuer adipiscing elit. Tempor a proin quam cursus
tincidunt id fringilla. Netus hac suscipit maximus hendrerit luctus massa.
Hendrerit mollis nibh curabitur magnis netus ad diam. Sapien per conubia id
primis auctor. Porta metus feugiat tincidunt amet suscipit habitasse.

%------------------------------------------------------------------------

\section{Conclusion and Future Work}\label{sec:conclusion-and-future-work}

Lorem ipsum dolor sit amet, consectetur adipiscing elit, sed do eiusmod tempor
incididunt ut labore et dolore magna aliqua. Ut enim ad minim veniam, quis
nostrud exercitation ullamco laboris nisi ut aliquip ex ea commodo consequat.
Duis aute irure dolor in reprehenderit in voluptate velit esse cillum dolore eu
fugiat nulla pariatur. Excepteur sint occaecat cupidatat non proident, sunt in
culpa qui officia deserunt mollit anim id est laborum.

%------------------------------------------------------------------------

\subsection{Summary}

Lorem ipsum odor amet, consectetuer adipiscing elit. Tempor a proin quam cursus
tincidunt id fringilla. Netus hac suscipit maximus hendrerit luctus massa.
Hendrerit mollis nibh curabitur magnis netus ad diam. Sapien per conubia id
primis auctor. Porta metus feugiat tincidunt amet suscipit habitasse.

    %------------------------------------------------------------------------

    {\small
        \bibliographystyle{ieee}
        \bibliography{kembuegler_bib}
    }

\end{document}
